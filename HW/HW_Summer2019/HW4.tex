\documentclass[]{article}
\usepackage{lmodern}
\usepackage{amssymb,amsmath}
\usepackage{ifxetex,ifluatex}
\usepackage{fixltx2e} % provides \textsubscript
\ifnum 0\ifxetex 1\fi\ifluatex 1\fi=0 % if pdftex
  \usepackage[T1]{fontenc}
  \usepackage[utf8]{inputenc}
\else % if luatex or xelatex
  \ifxetex
    \usepackage{mathspec}
  \else
    \usepackage{fontspec}
  \fi
  \defaultfontfeatures{Ligatures=TeX,Scale=MatchLowercase}
\fi
% use upquote if available, for straight quotes in verbatim environments
\IfFileExists{upquote.sty}{\usepackage{upquote}}{}
% use microtype if available
\IfFileExists{microtype.sty}{%
\usepackage{microtype}
\UseMicrotypeSet[protrusion]{basicmath} % disable protrusion for tt fonts
}{}
\usepackage[margin=1in]{geometry}
\usepackage{hyperref}
\PassOptionsToPackage{usenames,dvipsnames}{color} % color is loaded by hyperref
\hypersetup{unicode=true,
            colorlinks=true,
            linkcolor=Maroon,
            citecolor=Blue,
            urlcolor=blue,
            breaklinks=true}
\urlstyle{same}  % don't use monospace font for urls
\usepackage{color}
\usepackage{fancyvrb}
\newcommand{\VerbBar}{|}
\newcommand{\VERB}{\Verb[commandchars=\\\{\}]}
\DefineVerbatimEnvironment{Highlighting}{Verbatim}{commandchars=\\\{\}}
% Add ',fontsize=\small' for more characters per line
\usepackage{framed}
\definecolor{shadecolor}{RGB}{248,248,248}
\newenvironment{Shaded}{\begin{snugshade}}{\end{snugshade}}
\newcommand{\KeywordTok}[1]{\textcolor[rgb]{0.13,0.29,0.53}{\textbf{#1}}}
\newcommand{\DataTypeTok}[1]{\textcolor[rgb]{0.13,0.29,0.53}{#1}}
\newcommand{\DecValTok}[1]{\textcolor[rgb]{0.00,0.00,0.81}{#1}}
\newcommand{\BaseNTok}[1]{\textcolor[rgb]{0.00,0.00,0.81}{#1}}
\newcommand{\FloatTok}[1]{\textcolor[rgb]{0.00,0.00,0.81}{#1}}
\newcommand{\ConstantTok}[1]{\textcolor[rgb]{0.00,0.00,0.00}{#1}}
\newcommand{\CharTok}[1]{\textcolor[rgb]{0.31,0.60,0.02}{#1}}
\newcommand{\SpecialCharTok}[1]{\textcolor[rgb]{0.00,0.00,0.00}{#1}}
\newcommand{\StringTok}[1]{\textcolor[rgb]{0.31,0.60,0.02}{#1}}
\newcommand{\VerbatimStringTok}[1]{\textcolor[rgb]{0.31,0.60,0.02}{#1}}
\newcommand{\SpecialStringTok}[1]{\textcolor[rgb]{0.31,0.60,0.02}{#1}}
\newcommand{\ImportTok}[1]{#1}
\newcommand{\CommentTok}[1]{\textcolor[rgb]{0.56,0.35,0.01}{\textit{#1}}}
\newcommand{\DocumentationTok}[1]{\textcolor[rgb]{0.56,0.35,0.01}{\textbf{\textit{#1}}}}
\newcommand{\AnnotationTok}[1]{\textcolor[rgb]{0.56,0.35,0.01}{\textbf{\textit{#1}}}}
\newcommand{\CommentVarTok}[1]{\textcolor[rgb]{0.56,0.35,0.01}{\textbf{\textit{#1}}}}
\newcommand{\OtherTok}[1]{\textcolor[rgb]{0.56,0.35,0.01}{#1}}
\newcommand{\FunctionTok}[1]{\textcolor[rgb]{0.00,0.00,0.00}{#1}}
\newcommand{\VariableTok}[1]{\textcolor[rgb]{0.00,0.00,0.00}{#1}}
\newcommand{\ControlFlowTok}[1]{\textcolor[rgb]{0.13,0.29,0.53}{\textbf{#1}}}
\newcommand{\OperatorTok}[1]{\textcolor[rgb]{0.81,0.36,0.00}{\textbf{#1}}}
\newcommand{\BuiltInTok}[1]{#1}
\newcommand{\ExtensionTok}[1]{#1}
\newcommand{\PreprocessorTok}[1]{\textcolor[rgb]{0.56,0.35,0.01}{\textit{#1}}}
\newcommand{\AttributeTok}[1]{\textcolor[rgb]{0.77,0.63,0.00}{#1}}
\newcommand{\RegionMarkerTok}[1]{#1}
\newcommand{\InformationTok}[1]{\textcolor[rgb]{0.56,0.35,0.01}{\textbf{\textit{#1}}}}
\newcommand{\WarningTok}[1]{\textcolor[rgb]{0.56,0.35,0.01}{\textbf{\textit{#1}}}}
\newcommand{\AlertTok}[1]{\textcolor[rgb]{0.94,0.16,0.16}{#1}}
\newcommand{\ErrorTok}[1]{\textcolor[rgb]{0.64,0.00,0.00}{\textbf{#1}}}
\newcommand{\NormalTok}[1]{#1}
\usepackage{longtable,booktabs}
\usepackage{graphicx,grffile}
\makeatletter
\def\maxwidth{\ifdim\Gin@nat@width>\linewidth\linewidth\else\Gin@nat@width\fi}
\def\maxheight{\ifdim\Gin@nat@height>\textheight\textheight\else\Gin@nat@height\fi}
\makeatother
% Scale images if necessary, so that they will not overflow the page
% margins by default, and it is still possible to overwrite the defaults
% using explicit options in \includegraphics[width, height, ...]{}
\setkeys{Gin}{width=\maxwidth,height=\maxheight,keepaspectratio}
\IfFileExists{parskip.sty}{%
\usepackage{parskip}
}{% else
\setlength{\parindent}{0pt}
\setlength{\parskip}{6pt plus 2pt minus 1pt}
}
\setlength{\emergencystretch}{3em}  % prevent overfull lines
\providecommand{\tightlist}{%
  \setlength{\itemsep}{0pt}\setlength{\parskip}{0pt}}
\setcounter{secnumdepth}{0}
% Redefines (sub)paragraphs to behave more like sections
\ifx\paragraph\undefined\else
\let\oldparagraph\paragraph
\renewcommand{\paragraph}[1]{\oldparagraph{#1}\mbox{}}
\fi
\ifx\subparagraph\undefined\else
\let\oldsubparagraph\subparagraph
\renewcommand{\subparagraph}[1]{\oldsubparagraph{#1}\mbox{}}
\fi

%%% Use protect on footnotes to avoid problems with footnotes in titles
\let\rmarkdownfootnote\footnote%
\def\footnote{\protect\rmarkdownfootnote}

%%% Change title format to be more compact
\usepackage{titling}

% Create subtitle command for use in maketitle
\newcommand{\subtitle}[1]{
  \posttitle{
    \begin{center}\large#1\end{center}
    }
}

\setlength{\droptitle}{-2em}

  \title{}
    \pretitle{\vspace{\droptitle}}
  \posttitle{}
    \author{}
    \preauthor{}\postauthor{}
      \predate{\centering\large\emph}
  \postdate{\par}
    \date{5/15/2019}

\usepackage[table,xcdraw]{xcolor}

\begin{document}

\section{CS7290 Causal Modeling in Machine Learning: Homework
4}\label{cs7290-causal-modeling-in-machine-learning-homework-4}

\subsection{Submission guidelines}\label{submission-guidelines}

Use a Jupyter notebook and/or R Markdown file to combine code and text
answers. Compile your solution to a static PDF document(s). Submit both
the compiled PDF and source files. The TA's will recompile your
solutions, and a failing grade will be assigned if the document fails to
recompile due to bugs in the code. If you use
\href{https://colab.research.google.com/notebook}{Google Collab}, send
the link as well as downloaded PDF and source files.

\subsection{Background}\label{background}

This assignment is going to cover several topics, including some that
haven't been taught at the time this was assigned. We will cover those
topics in subsequent classes.

\begin{itemize}
\tightlist
\item
  Recognizing valid adjustment sets
\item
  Covariate adjustment with parent and back-door criterion
\item
  Front-door criterion
\item
  Propensity matching and inverse probability weighting
\item
  Intro to structural causal models
\end{itemize}

\subsection{1.}\label{section}

Consider the following data comparing purchases on an e-commerce website
with high and low exposure to promotions.

\begin{longtable}[]{@{}lll@{}}
\toprule
& Promotional Exposure &\tabularnewline
\midrule
\endhead
& High (X=1) & Low (X=0)\tabularnewline
Purchase (Y = 1) & 30 & 16\tabularnewline
No Purchase (Y = 0) & 691.30 & 590.10\tabularnewline
\bottomrule
\end{longtable}

P(X = 1) = 0.5433931 P(Y = 1) = 0.03465421 Given these data, we wish to
estimate the probabilities that high exposure to promotions was a
necessary (or sufficient, or both) cause of purchase. We assume
monotonicity -- exposure to promotions didn't cause anyone NOT to make a
purchase.

We assume the following DAG. Here, the vari

\includegraphics{HW4_files/figure-latex/xqy-1.pdf}

We assume that purchases \(Y\) has the following simple disjunctive
mechanism:

\begin{align} 
\mathbf{M} =
\left\{\begin{matrix}
n_x &\sim &\text{BernoulliBool}(p=0.5433931)\\ 
n_q &\sim &\text{BernoulliBool}(p=0.05)\\ 
n_z &\sim &\text{BernoulliBool}(p=.0077)\\ 
x &= &n_x \\ 
q &= &n_q \\
y &= &(x \wedge q) \vee n_y
\\ 
\end{matrix}\right. \nonumber
\end{align}

\subsection{\texorpdfstring{Calculate the probability of necessity:
\(P(Y_{0} = 0| X = 1, Y = 1)\)}{Calculate the probability of necessity: P(Y\_\{0\} = 0\textbar{} X = 1, Y = 1)}}\label{calculate-the-probability-of-necessity-py_0-0-x-1-y-1}

\begin{enumerate}
\def\labelenumi{\arabic{enumi}.}
\tightlist
\item
  First infer the distribution of \(n_y\) and \(n_q\) give X = 1 and Y =
  1.
\end{enumerate}

Solution for (1):

\(P(Y_{0} = 0| X = 1, Y = 1) = \sum_q P(Y_{x'} = y'| X = x, Y = y, Q = q)P(Q = q)\)

Make a joint probability table:

\begin{longtable}[]{@{}lllllll@{}}
\toprule
\(n_x\) & \(n_q\) & \(n_y\) & x & q & y & prob\tabularnewline
\midrule
\endhead
0 & 0 & 0 & 0 & 0 & 0 & 0.43\tabularnewline
0 & 0 & 1 & 0 & 0 & 1 & 0.003\tabularnewline
0 & 1 & 0 & 0 & 1 & 1 & 0.023\tabularnewline
0 & 1 & 1 & 0 & 1 & 1 & 2\times 10\^{}\{-4\}\tabularnewline
1 & 0 & 0 & 1 & 0 & 1 & 0.512\tabularnewline
1 & 0 & 1 & 1 & 0 & 1 & 0.004\tabularnewline
1 & 1 & 0 & 1 & 1 & 1 & 0.003\tabularnewline
1 & 1 & 1 & 1 & 1 & 1 & 2\times 10\^{}\{-4\}\tabularnewline
\bottomrule
\end{longtable}

\begin{Shaded}
\begin{Highlighting}[]
\NormalTok{y_prob <-}\StringTok{ }\ControlFlowTok{function}\NormalTok{(ny_p, nq_p, }\DataTypeTok{n=}\DecValTok{1000000}\NormalTok{)\{}
\NormalTok{    nx =}\StringTok{ }\KeywordTok{rbinom}\NormalTok{(n, }\DecValTok{1}\NormalTok{, }\FloatTok{0.5433931}\NormalTok{)}
\NormalTok{    nq =}\StringTok{ }\KeywordTok{rbinom}\NormalTok{(n, }\DecValTok{1}\NormalTok{, nq_p)}
\NormalTok{    ny =}\StringTok{ }\KeywordTok{rbinom}\NormalTok{(n, }\DecValTok{1}\NormalTok{, ny_p)}
\NormalTok{    x =}\StringTok{ }\NormalTok{nx}
\NormalTok{    q =}\StringTok{ }\NormalTok{nq}
\NormalTok{    y =}\StringTok{ }\NormalTok{((x }\OperatorTok{*}\StringTok{ }\NormalTok{q) }\OperatorTok{+}\StringTok{ }\NormalTok{ny }\OperatorTok{>}\StringTok{ }\DecValTok{0}\NormalTok{)}
    \KeywordTok{sum}\NormalTok{(y)}\OperatorTok{/}\NormalTok{n}
\NormalTok{\}}
\end{Highlighting}
\end{Shaded}


\end{document}
